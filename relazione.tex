\documentclass[a4paper]{article}
\usepackage{graphicx} % Required for inserting images
\usepackage[font=it, width=0.9\linewidth]{caption}
\usepackage{mathtools}
\usepackage{listings}
\usepackage{graphicx}
\usepackage{bm}
\usepackage{caption}
\usepackage{float}
\usepackage{svg}
\usepackage{hyperref}
\usepackage{longtable}
\renewcommand{\figurename}{Fig.}
\renewcommand{\tablename}{Tab.}
\newcommand{\equationname}{Eq.}
\title{Misura della caratteristica di uscita di un BJT P-N-P
in configurazione a emettitore comune}
\author{Bellini Samuele, Caprara Francesco}
\date{Dicembre 2025, turno 4, tavolo 11}

\begin{document}

\maketitle

\section{Abstract}
\section{Introduzione}
\section{Apparato sperimentale e svolgimento}
Gli strumenti di misura che abbiamo utilizzato sono un oscilloscopio analogico GW~GOS-652G 50Mhz e un multimetro digitale FLUKE~175. 
L'oscilloscopio ha una portata di 400~V, divisa in 12 sezioni definite dal valore scelto per i volt associati ad ognuna delle 8 divisioni dello schermo.
I valori disponibili da associare alle divisioni sono:
\begin{table}[H]
    \centering
    \setlength{\tabcolsep}{4pt}  %column separation%
    % \renewcommand{\arraystretch}{1.25} %row separation%
    \begin{tabular}{c  c  c  c  c  c  c  c  c  c  c  c  c}
        \textbf{Volts/Div} 0.001 & 0.002 & 0.005 & 0.01 & 0.02 & 0.05 & 0.1 & 0.2 & 0.5 & 1 & 2 & 5 \\
    \end{tabular}
\end{table}
\noindent L'errore associato alle misure fatte con l'oscilloscopio è dato dalla formula
\[\sigma = \sqrt{{\sigma_L} ^2 + {\sigma_Z}^2 + {\sigma_C}^2 } \]
dove \(\sigma_L\) è l'errore sulla lettura, pari a \(\frac{1}{2} \frac{Volts/Div}{5}\), in quanto ogni divisione è composta da 5 tacchette e tutti i segnali misurati sono sufficientemente puliti da poter apprezzare la mezza tacchetta,
\(\sigma_Z\) è l'errore sullo zero, cioè l'errore dovuto all'imprecisione nel posizionare la linea di 0~V, ed è
pari alla \(\sigma_L\) associata agli 0.001~Volts/Div, in quanto abbiamo utilizzato 
quella scala per posizionare lo zero, e \(\sigma_C\) è l'errore dichiarato dal costruttore, pari al 3\% della misura. 
\\Il multimetro è stato utilizzato come amperometro per misurare correnti continue; in questa modalià
ha una portata di 10~A. Nell'intervallo [-60, 0]mA, dove sono presenti tutte le nostre misure,
l'errore da associare alla misura è dato da 
\[\sigma = { 1.5\% \, \left| misura \right| + 0.03\,mA } \]
dove \(misura\) è il valore letto dal multimetro in mA.
Abbiamo inoltre utilizzato il generatore da banco TTi EB2025T per ottenere una tensione costante di -5~V. 


Per misurare due curve caratteristiche del BJT a emettitore comune abbiamo inizialmente realizzato
il circuito in \figurename~\ref{App_sper:circuiti} a sinistra, variando il potenziometro collegato alla 
base fino ad ottenere una lettura di corrente passante per la base pari a \(-100\mu A\). Abbiamo poi
sostituito l'amperometro con un filo, mantenendo il valore del potenziometro fissato, in modo da lasciare 
invariata \(I_{B}\), e abbiamo realizzato il circuito in \figurename~\ref{App_sper:circuiti} a destra.
In questa configurazione è possibile utilizzare il potenziometro collegato al collettore per modificare sia 
\(I_{C}\) sia \(V_{CE}\); Misurando la prima con l'amperometro e la seconda con l'oscilloscopio abbiamo
ottenuto 32 punti, nell'intervallo di tensioni [0.05, 4]V.
Abbiamo poi ripetuto lo stesso procedimento fissando la corrente di base a \(50\mu A\), ottenendo un altro set
da 32 punti I-V.\\
Per le misure è stato necessario fissare prima la tensione, aspettare che la corrente misurata dall'amperometro 
si stabilizzasse per qualche secondo e prendere come valore di corrente l'ultimo letto, in quanto durante i primi istanti
dopo aver cambiato valore di tensione, in particolare per le tensioni più alte, la corrente saliva troppo velocemente 
per poter definire bene quale fosse il valore corretto.

\begin{figure}[H]
    \centering
    \includegraphics[width=.9\linewidth]{Immagini/circuiti.png}
    \caption{ \label{App_sper:circuiti}}
\end{figure}

\section{Risultati e discussione}
\section{Conclusioni}
\section{Appendice}
\end{document}
